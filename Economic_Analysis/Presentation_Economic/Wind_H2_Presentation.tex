\documentclass[aspectratio=169, 10pt]{beamer}

% --- THEME SETTINGS ---
\usetheme{Madrid} % Clean, professional theme
\usecolortheme{whale} % Blue/Dark color scheme fitting for energy
\usefonttheme{professionalfonts}

% --- PACKAGES ---
\usepackage{graphicx}
\usepackage{booktabs} % For nicer tables
\usepackage{tikz}
\usepackage{amsmath}
\usepackage{eurosym} % For Euro symbol

% --- METADATA ---
\title[Hydrogen Arbitrage Optimization]{Optimizing Wind Energy Revenue through Hydrogen Arbitrage}
\subtitle{A Techno-Economic Analysis of a Hybrid 50MW Electrolyzer System in Germany}

% --- SLIDE CONTENT ---
\begin{document}
	
	% Slide 1: Title
	\begin{frame}
		\titlepage
	\end{frame}
	
	% Slide 2: The Problem
	\begin{frame}{The Challenge: Cannibalization \& Curtailment}
		\begin{columns}
			\column{0.6\textwidth}
			\begin{itemize}
				\item \textbf{Price Cannibalization:} As renewable penetration rises, market prices drop during high-wind hours.
				\item \textbf{Grid Congestion:} Northern Germany faces significant grid bottlenecks.
				\item \textbf{Curtailment:} Wind farms are frequently turned off, wasting potential green energy.
			\end{itemize}
			\vspace{0.5cm}
			\textbf{Research Question:} \\
			\textit{Can a hybrid Hydrogen system monetize this "lost" energy while remaining profitable without massive subsidies?}
			
			\column{0.4\textwidth}
			\begin{block}{Data Insight}
				Analysis confirms a divergence between \textbf{Market Average Price} and the realized \textbf{Capture Price} for wind assets [1].
			\end{block}
		\end{columns}
	\end{frame}
	
	% Slide 3: System Design
	\begin{frame}{System Design \& Technical Parameters}
		\begin{table}[]
			\centering
			\begin{tabular}{l l}
				\toprule
				\textbf{Parameter} & \textbf{Value / Description} \\
				\midrule
				\textbf{Wind Asset} & 210 MW Onshore Capacity \\
				Location & Reußenköge, Germany (ERA5 Data) [3] \\
				\midrule
				\textbf{Electrolyzer} & 50 MW PEM System \\
				Model & Siemens Energy Silyzer 300 [4] \\
				Efficiency & 52.2 kWh / kg H$_2$ (Nominal) \\
				Min. Load & 10\% (5 MW) \\
				\bottomrule
			\end{tabular}
		\end{table}
		
		\vspace{0.3cm}
		\centering
		\textbf{Configuration:} Co-located system ("Behind-the-Meter") to minimize grid fees.
	\end{frame}
	
	% Slide 4: Methodology
	\begin{frame}{Methodology: The "Smart Arbitrage" Logic}
		\textbf{Objective:} Maximize Net Present Value (NPV) via dynamic dispatch.
		
		\vspace{0.5cm}
		\begin{alertblock}{The Switching Threshold}
			We calculate a dynamic Breakeven Price ($P_{switch}$) to decide operation modes:
			\[
			P_{switch} = \frac{1000}{\eta_{el}} \times P_{H2}
			\]
			Where $\eta_{el} = 52.2$ kWh/kg and $P_{H2} = \text{\euro}5.00$/kg.
		\end{alertblock}
		
		\vspace{0.5cm}
		\textbf{The Decision Rule:}
		\begin{itemize}
			\item If Grid Price $<$ \textbf{\euro95.79 / MWh} $\rightarrow$ \textbf{\color{green} Produce Hydrogen}
			\item If Grid Price $>$ \textbf{\euro95.79 / MWh} $\rightarrow$ \textbf{\color{blue} Sell Electricity}
		\end{itemize}
	\end{frame}
	
	% Slide 5: Data Sources
	\begin{frame}{Data Sources \& Validation}
		\begin{itemize}
			\item \textbf{Electricity Prices:} Day-Ahead Auction prices (DE-LU Zone, 2024) [1].
			\item \textbf{Wind Resource:} ERA5 Reanalysis data (ECMWF), processed via Xarray [3].
			\item \textbf{Grid Stress:} Historical Redispatch 2.0 data (TenneT TSO) [2].
			\item \textbf{Financials:} CAPEX (\euro1,800/kW) and WACC (7\%) aligned with IEA Global Hydrogen Review [6].
		\end{itemize}
	\end{frame}
	
	% Slide 6: Simulation Results
	\begin{frame}{Simulation Results: Financial Viability}
		\begin{columns}
			\column{0.5\textwidth}
			\textbf{Revenue Uplift}
			\begin{itemize}
				\item Hybrid system generated \textbf{\euro10.96 Million} in \textit{extra} annual revenue vs. baseline.
				\item Driven by avoiding low/negative price hours.
			\end{itemize}
			
			\vspace{0.3cm}
			\textbf{Cost Competitiveness}
			\begin{itemize}
				\item \textbf{LCOH:} \euro4.87 / kg
				\item \textbf{Sales Price:} \euro5.00 / kg
				\item \textbf{Margin:} \euro0.13 / kg
			\end{itemize}
			
			\column{0.5\textwidth}
			\begin{block}{Final Investment Verdict}
				\centering
				\vspace{0.2cm}
				\Large \textbf{APPROVED} \\
				\vspace{0.2cm}
				\normalsize
				\begin{tabular}{l r}
					\textbf{NPV:} & \textbf{\euro7.04 Million} \\
					\textbf{IRR:} & \textbf{8.0\%} \\
				\end{tabular}
				\vspace{0.2cm}
				
				\textit{\small Project clears the 7\% WACC hurdle.}
			\end{block}
		\end{columns}
	\end{frame}
	
	% Slide 7: Visualization
	\begin{frame}{Revenue Diversification}
		\centering
		% REPLACE 'Hybrid_Stacked_vs_Base.png' with your actual filename
		% Ensure the image is in the same folder as the .tex file
		\includegraphics[width=0.85\textwidth, height=0.65\textheight, keepaspectratio]{Hybrid_Stacked_vs_Base.png}
		
		\vspace{0.2cm}
		\small \textit{Figure: Monthly revenue comparison showing Hydrogen (Green) as a revenue floor during summer months.}
	\end{frame}
	
	% Slide 8: Sensitivity Analysis
	\begin{frame}{Sensitivity \& Risk Analysis}
		\textbf{Critical Risk Factor: Hydrogen Offtake Price}
		
		\vspace{0.5cm}
		\begin{itemize}
			\item The project operates on a thin margin (\euro0.13/kg).
			\item A price drop of just \textbf{\euro0.15/kg} (to \euro4.85) turns NPV negative.
		\end{itemize}
		
		\vspace{0.5cm}
		\textbf{Mitigation Strategy:}
		\begin{enumerate}
			\item \textbf{Long-Term PPA:} Securing a fixed-price offtake agreement $ > \text{\euro}5.00$/kg is mandatory for FID.
			\item \textbf{Grid Services:} Future qualification for Frequency Containment Reserve (FCR) could provide ancillary revenue.
		\end{enumerate}
	\end{frame}
	
	% Slide 9: Conclusion
	\begin{frame}{Conclusion}
		\begin{enumerate}
			\item \textbf{Feasibility Confirmed:} A 50MW Electrolyzer co-located with 210MW wind is economically viable in Germany (8.0\% IRR).
			\item \textbf{Value of Flexibility:} The dynamic switching logic (curtailing H$_2$ when power $>$ \euro96/MWh) is critical to profitability.
			\item \textbf{Grid Impact:} The system acts as a flexible load, reducing physical curtailment and aiding grid stability.
		\end{enumerate}
		
		\vspace{1cm}
		\centering
		\large \textbf{Thank You! Questions?}
	\end{frame}
	
	% Slide 10: References
	\begin{frame}[allowframebreaks]{References}
		\tiny
		\setbeamertemplate{bibliography item}[text]
		\begin{thebibliography}{9}
			
			\bibitem{energycharts}
			Fraunhofer ISE. (2024). \textit{Energy-Charts: Public Net Electricity Generation in Germany}.
			
			\bibitem{tennet}
			TenneT TSO. (2024). \textit{Redispatch 2.0 Data Downloads}.
			
			\bibitem{era5}
			ECMWF. (2024). \textit{ERA5 Reanalysis Data}. Copernicus Climate Change Service.
			
			\bibitem{siemens}
			Siemens Energy. (2023). \textit{Silyzer 300 PEM Electrolyzer Datasheet}.
			
			\bibitem{eex}
			EEX. (2024). \textit{HYDRIX - European Green Hydrogen Index}.
			
			\bibitem{iea}
			IEA. (2024). \textit{Global Hydrogen Review 2024}. International Energy Agency, Paris.
			
		\end{thebibliography}
	\end{frame}
	
\end{document}